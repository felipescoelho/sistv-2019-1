\documentclass[11pt]{article}

%4pp das transps
% psnup -W415 -H273 -s.9 -b10 -d10 -c -4 05-Tecnicas_Compressao_Imagens.ps 05-Tecnicas_Compressao_Imagens.4pp.ps

\usepackage[utf8]{inputenc} % PARA USAR PALAVRAS EM PORT
%\usepackage[portuguese]{babel}
\usepackage{a4wide}
\usepackage{color}
\usepackage{colordvi}
\usepackage{epsf,epsfig}
\usepackage{amsmath}
\usepackage{amsfonts}
\usepackage{enumerate}

\begin{document} 

\title{Lista de Exercícios de Sistemas de TV -- 2019-1}
\author{Lisandro Lovisolo \\ lisandro@uerj.br \\ PROSAICO -- DETEL -- UERJ \\ \begin{small} Laboratório de Processamento de Sinais, Aplicações Inteligentes e Comunicações \end{small} \\ Departamento de Engenharia Eletrônica e Telecomunicações \\ Universidade do Estado do Rio de Janeiro}

\maketitle



\section{Mascaramento de Áudio}

\textbf{Objetivo:} Avaliar empiricamente o mascaramento de áudio, e desenvolver capacidades de implementação de funções e uso de comparações na escala de decibéis. Objetivamos ainda avaliar nossa capacidade auditiva. 

\subsection{Mascaramento Simples}

\begin{enumerate}
\item \textbf{Tarefa:} Faça um programa \textsf{MatLab} que some dois tons de áudio. Isto é, gere dois sinais $A_1 \cos(2\pi f_1) + A_2 \cos(2\pi f_2)$, sendo $f_i$ e $A_i$ parâmetros de entrada da função e que toque (no alto-falante) esse sinal.   

\item \textbf{Tarefa:} Explique o raciocínio e os testes realizados para a consecução exitosa da tarefa acima.

\item \textbf{Tarefa:} Utilize essa função para avaliar o mascaramento auditivo de um tom de 800Hz por outro de 1000Hz.

\item \textbf{Tarefa:} Discuta o resultado obtido. Compare as potências dos sinais resultantes na saída. Avalie as potência usando db. 
\end{enumerate}

\subsection{Mascaramento em Função da Frequência}

\begin{enumerate}

\item \textbf{Tarefa:} Utilize a metodologia acima para avaliar o mascaramento auditivo de um tom de $f$ Hz por outro de 1.1 $f$ Hz. Gere um gráfico que apresente a razão mínima entre as potências do sinal de menor frequência sobre o de maior que faz com que o tom de maior frequência esteja mascarado (seja inaudível). Faça isso usando frequências de teste em oitavas (por exemplo, 110, 220, 440, 880, 1760, 3520, etc).

\end{enumerate}

\subsection{Mascaramento de Frequências Relativas}

\begin{enumerate}

\item \textbf{Tarefa:} Utilize a metodologia acima para avaliar o mascaramento auditivo de um tom de $f$ Hz por outro de   $(1+x)f$ Hz, $x\in[-1/2,1/2]$ . Gere um gráfico que apresente a razão mínima entre as potências do sinal de frequência $f$ sobre o de frequência $(1+x)f$ que faz com que o tom de frequência $(1+x)f$ esteja mascarado (seja inaudível). 

\end{enumerate}
\end{document}