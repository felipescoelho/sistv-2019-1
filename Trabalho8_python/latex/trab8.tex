\documentclass[11pt]{article}

%4pp das transps
% psnup -W415 -H273 -s.9 -b10 -d10 -c -4 05-Tecnicas_Compressao_Imagens.ps 05-Tecnicas_Compressao_Imagens.4pp.ps

\usepackage[utf8]{inputenc} % PARA USAR PALAVRAS EM PORT
%\usepackage[portuguese]{babel}
\usepackage{a4wide}
\usepackage{color}
\usepackage{colordvi}
\usepackage{epsf,epsfig}
\usepackage{amsmath}
\usepackage{amsfonts}
\usepackage{enumerate}

\begin{document} 

\title{Lista de Exercícios de Sistemas de TV -- 2019-1}
\author{Lisandro Lovisolo \\ lisandro@uerj.br \\ PROSAICO -- DETEL -- UERJ \\ \begin{small} Laboratório de Processamento de Sinais, Aplicações Inteligentes e Comunicações \end{small} \\ Departamento de Engenharia Eletrônica e Telecomunicações \\ Universidade do Estado do Rio de Janeiro}

\maketitle

\section{Avaliação Objetiva de Qualidade de Imagens}

\textbf{Objetivo:} O principal objetivo das atividades desta seção é reforçar conhecimentos sobre a avaliação objetiva de vídeo e imagens digitais. O secundário é expandir os conhecimentos sobre manipulação de matrizes e exibição de imagens usando o \textsf{Matlab}.

\subsection{Métricas Tradicionais}

\begin{enumerate}

\item \textbf{Tarefa:} Avalie as diferentes interpolações implementadas no item 6.1.3 em função de métricas de qualidade definidas abaixo. Sejam:
\begin{align}
i(c,l) & \textrm{ a luminância do pixel da imagem original na posição } c,l \\
\hat{i}(c,l) & \textrm{ a luminância do pixel da imagem ''reconstruída'' na posição } c,l \\
d(c,l) & = i(c,l) - \hat{i}(c,l).
\end{align}
Podemos definir as métricas
\begin{align}
\textsf{MSAD} & = \frac{1}{CL}\sum_c{\sum_l{ \left | d(c,l) \right |  }}, \\
\textsf{MSE} & = \frac{1}{CL}\sum_c{\sum_l{ \left | d(c,l) \right |^2  }}, \\
\textsf{PSNR} & = 10 \log{\left ( \frac{255^2}{\textsf{MSE}} \right )}, 
\end{align}
onde $C$ é a quantidade de colunas da imagem e $L$ a quantidade de linhas.

\begin{itemize}
\item[\textit{Dica}:] Empregue a função \textsf{MatLab} \textsf{mean} para os cálculos das médias consideradas acima. 

\item[\textit{Dica}:] Para realizar as avaliações:
\begin{enumerate}
\item Pegue uma imagem de resolução elevada e subamostre-a de 2 e 4 em cada direção (já vimos forma de fazê-lo), assim temos três imagens: $I$ -- a original, $I_2$ -- a subamostrada de 2 e $I_4$  -- a subamostrada de 4. 
\item Usando o interpoladores desenvolvidos e discutidos em itens anteriores interpole as imagens subamostradas de forma a obter versões interpoladas com as mesmas dimensões da imagem original $I$.
\end{enumerate}
\end{itemize}

\item \textbf{Tarefa:} Deixe claro quais os métodos de subamostragem e interpolação empregados.

\item \textbf{Tarefa:} Apesente uma tabela com as métricas de qualidade descritas acima, para as diferentes versões obtidas em resolução original da imagem alvo. 

\item \textbf{Pergunta:} Observando os valores obtidos acima e as qualidades subjetivas observadas (por vocês) nas imagens, podemos afirmar alguma coisa? O que? Explique e justifique.

\end{enumerate}

\subsection{SSIM}

\begin{enumerate}

\item \textbf{Tarefa:} Uma métrica que tem se tornado cada vez mais utilizada para a avaliação da qualidade de imagens e vídeo é o SSIM (\emph{Strcutural Similarity Index Metric}. Documentação e implementação da mesma pode ser encontrada facilmente na internet e %em \textsf{https://ece.uwaterloo.ca/ \~{ }z70wang/research/ssim/}, que 
segue anexa a este documento. Explique o princípio de funcionamento do SSIM.

\item \textbf{Tarefa:} Repita os experimentos da subseção anterior com a implementação do SSIM disponibilizada. Isto é expanda a tabela acima obtida para incluir o índice de avaliação fornecido pelo SSIM.

\item \textbf{Pergunta:} Observando os valores obtidos acima e as qualidades subjetivas observadas (por vocês) nas imagens, podemos afirmar alguma coisa? O que? Explique e justifique.

\end{enumerate}

\end{document}


